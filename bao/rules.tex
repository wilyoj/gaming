\documentclass[12pt,a4paper]{article}
\usepackage[utf8]{inputenc}
\usepackage{amsmath}
\usepackage{amsfonts}
\usepackage{amssymb}
\usepackage{url}
\setcounter{secnumdepth}{5}
%\counterwithin{enumitem}{section}
\title{K.I.B.A. - Klubo Internacia de Bao-Amantoj\\\url{www.kibao.org}}
\date{}
\begin{document}
\maketitle
\section{General Rules}\label{GR}
\subsection{Goal of the game}\label{Gotg} 
The goal of Bao la Kiswahili is to empty the front row of the opponent or to make it impossible for the opponent to move.
\subsection{End of the game}\label{Eotg} 
The game ends in one of the following situations:
\subsubsection{}\label{Eotg1}
The front row of a player is empty (even before his/her move ends) or;
\subsubsection{}\label{Eotg2}
A player cannot move.

\subsection{Sowing}\label{Sow} 
The basic move of Bao is sowing of seeds.  Sowing is the process of putting at least two seeds one by one in consecutive pits in the own field, consisting of the nearest two rows of the board, in clockwise or anti-clockwise direction.  During sowing, the direction of the sowing cannot change. Every sowing has a starting pit, at least two seeds to sow, a sowing direction, and an ending pit.
\subsubsection{}\label{Sow1.1}
During sowing the direction of sowing cannot change, unless the sowing ends with a harvest in a {\it kichwa} or {\it kimbi} (see \ref{Hc5}, \ref{Kek}).
\subsubsection{}\label{Sow1.2}
It is not allowed to start a sowing with a single seed.

\subsection{Harvesting (capture)}\label{Hc} 
Capture in Bao is harvesting in the opponent field. Harvesting is only allowed if a sowing ends in a non-empty pit (e.g., after sowing there are more than one seed) at the front row that has an opposing non-empty (with at least one seed) pit at the front row of the opponent; these seeds are called mtaji.

\subsubsection{}\label{Hc1}
If a player can harvest, he must do so.

\subsubsection{}\label{Hc2}
The harvest in Bao consists of taking all seeds out of the harvested pit in the opponent's front row and sowing them immediately at the own side of the board. The sowing of harvested seeds must start at one of the {\it kichwa's}.

\subsubsection{}\label{Hc3}
The existing sowing direction of the move must be sustained. At the start of beginning of a move in {\it kunamua} stage the direction can be chosen by the player.

\subsubsection{}\label{Hc4}
If the sowing starts at the left {\it kichwa}, the sowing direction is clockwise, if the sowing start at the right {\it kichwa}, the sowing direction is anti-clockwise.

\subsubsection{}\label{Hc5}
If the harvesting pit is the left {\it kichwa} or {\it kimbi}, the sowing must start at the left {\it kichwa}. If the harvesting pit is the right {\it kichwa} or {\it kimbi}, the sowing must start at the right {\it kichwa} (see \ref{Kek}).


\subsection{Move}\label{Mov} 
A move in Bao is a sequence of sowings and harvests by one player.

\subsubsection{}\label{Mov1}
A move must start with a harvest, if possible.
\subsubsection{}\label{Mov2}
If a move can't start with a harvest, it's possible to move from a pit with more than a single seed of the front row. Only if all the pits of the front row have less than two seeds it's possible to move from the back row.
\subsubsection{}\label{Mov3}
If a move does not start with a harvest, then harvesting is not allowed at all during that move. This move is called {\it kutakata}.
\paragraph{}\label{Mov3.1} 
If the only filled pit on the front row is one of the {\it kichwa}-s, then {\it kutakata} cannot be done in the direction of the back row (because the front row will be empty and the game is a loss, even if the move could end in the front row).
\subsubsection{}\label{Mov4} 
A move stops if a sowing ends in an empty pit, but may also stop at the {\it nyumba} (see \ref{NyuGR2}) or at a {\it kutakatia}-ed pit (see \ref{Kut2}).
\subsubsection{}\label{Mov5} 
{\bf Infinite moves.} A move can take a long time and sometimes last forever. However, these infinite moves are illegal. If no finite move is available, the game is lost by the player at move. Because infinity of a move can be very lengthy to prove, at the beginning of the game the number of repeated moves required before pronouncement of an ``infinite move'' is defined.

\subsection{Kuenedelea}\label{Kue} 
If a sowing ends in a non-empty pit (e.g., after sowing there is more than one seed in the ending pit) and a harvest is not allowed, then the move continues in the same direction by taking all the seeds from that pit and sowing the seeds starting at the next pit in the same direction. This continuation of sowing is called {\it kuendelea}.
\subsubsection{}\label{Kue1} 
{\it Kuendelea} stops if a harvest is possible. The move continues with the harvest. The direction of sowing of harvested seeds is the same as the direction of {\it kuenedelea}, unless harvest occurs at the {\it kichwa} or {\it kimbi} (see \ref{Kek}).
\subsubsection{}\label{Kue2} 
{\it Kuendelea} stops if the sowing ends at the owned {\it nyumba} that contains six or more seeds if the player does not decide (see \ref{NyuGR2}) to play the {\it nyumba} in general, or does not decide (see \ref{NyuTndks3.2}) to play it during {\it kunamua}.
\subsubsection{}\label{Kue3} 
{\it Kuendelea} stops if the sowing ends at a {\it kutakatia}-ed pit.

\subsection{Stages in Bao}\label{SiB}
There are two stages in Bao: {\it kunamua} (initial stage) and {\it mtaji} (final stage).

\section{{\it Kunamua} stage}\label{Kun}
\subsection{Start}\label{Sta}
The game starts in {\it kunamua} stage with the following board configuration: there are six seeds in South's {\it nyumba} and two seeds in the pit to the right of the {\it nyumba} and again two seeds in the next pit to the right.  The same number of seeds are in North's {\it nyumba} and in the consecutive pits to the right (of North). Each player has 22 seeds in store.

\subsubsection{}\label{Sta1}
The first player is South
\subsubsection{}\label{Sta2}
The player's move starts with a sowing of one seed taken from the store. Applying rule \ref{Hc1}. This is a the first sowing of the move and it has a direction yet.

\subsection{\bf Harvesting in {\it kunamua} stage.}\label{Hikuns} 
The player takes one seed from the store and starts the first sowing of the move (see \ref{Hc}).

\subsubsection{}\label{Hikuns1} 
A move must start with a harvest, if possible, so if a non-empty (with at least one seed) pit at the front row has an opposing non-empty pit at the front row of the opponent, the player has to put a seed into that pit, to take all seeds out of the opponent's pit and to sow them immediately at the own side of the board starting at one of the {\it kichwa}-s.
\subsubsection{}\label{Hikuns2}
If the harvesting pit is not a  {\it kichwa} or {\it kimbi}, the player may choose which {\it kichwa} to sow from, the player can choose the {\it kichwa} to start at.  Since the player has taken the seed from the store, the player doesn't have a sowing direction yet, so he/she can choose it at the second sowing.

\subsection{\bf Kutakata in {\it kunamua} stage}\label{Kikuns} 
If a player cannot harvest he must {\it kutakata}.  In {\it kunamua}, kutakata starts with sowing one seed from the store in a non-empty pit in the front row. {\it Kutakata} cannot start in the back row (see \ref{Mov2}).

\subsubsection{}\label{Kikuns1} 
If the only filled pit on the front row is one of the  {\it kichwa}-s, then {\it kutakata} cannot be done in the direction of the back row (because the front row will be empty and the game is a loss, see \ref{Mov3.1}).

\subsubsection{}\label{Kikuns2} 
{\it Kutakata} cannot start from the owned {\bf nymba} with six or more seeds unless it is the only filled pit in the front row. If it is the only pit filled in the front row, then one seed from the store must be put in it, two seeds have to be taken out and spread to the left or the right.

\subsubsection{}\label{Kikuns3}
If the {\bf nyumba} is lost and if there are pits with more than one seed on the front row {\it kutakata} cannot start from a pit with only one seed.

\section{{\it Mtaji} stage}\label{Ms}

\subsection{}\label{Ms1} 
The {\it mtaji} stage starts if all seeds are on the board.
\subsection{}\label{Ms2} 
If the {\bf nyumba} happens to be still owned by one of the players, it stays owned until the first harvest occurs. Only the general rules for the {\bf nyumba} apply. {\it Kutakatia} rule (see \ref{Kut1.1}) does apply on the {\bf nyumba}.
\subsection{}\label{Ms3}
Every move in {\it mtaji} stage starts with selecting a pit and sowing the content in a chosen direction (see \ref{Sow}).

\subsubsection{}\label{Ms3.1} 
In {\it mtaji}, stage only pits can be played that contain more than one seed (see \ref{Sow1.2}).
\subsubsection{}\label{Ms3.2} 
If both rows of a player contain pits with one seed or the front row only contains empty pits, this player loses the game (see \ref{Eotg}).

\subsection{\bf Harvest move in {\it mtaji} stage.}\label{Hmims} 
A move in {\it mtaji} stage must start from a pit on the front row or back row that contains more than one seed, whose seeds allow a harvest (see \ref{Mov1}).

\subsubsection{}\label{Hmims1}
The direction of the sowing of the harvested seeds is the same as the selected move direction unless harvest occurs on the {\it kichwa} or {\it kimbi} (see \ref{Kek}).
\subsubsection{}\label{Hmims2}
It's not allowed to harvest starting a move from a pit with more than 15 seeds, even if it is the {\it nyumba}.

\subsection{\bf Kutakata in {\it mtaji} stage}\label{Kims}
If no harvest move is possible, the player must {\it kutakata} from the front row (see \ref{Mov2}).

\subsubsection{}\label{Kims1} 
The player can start from the back row only if the front row only contains pits with one seed.
\subsubsection{}\label{Kims2}
If the only filled pit on the front row is one of the {\it kichwa}-s, kutakata cannot go in the direction of the back row (see \ref{Kikuns1}).

\section{Special pits}\label{Sp}

\subsection{\bf {\it Kichwa} e {\it kimbi}}\label{Kek}
If a sowing ends with a harvest in a {\it kichwa} or {\it kimbi} the sowing of the seeds must begin from the nearest {\it kichwa}, so the direction can change.
\subsection{\bf Nyumba}\label{Nyu}

\subsubsection{\bf General Rules}\label{NyuGR}

\paragraph{}\label{NyuGR1}
 Players loose their {\it nyumba} if it is emptied or after the first harvest from the {\it nyumba}.
\paragraph{}\label{NyuGR2}
 During a move started with a harvest if the sowing {\it kuendelea} ends at the owned {\it nyumba} that contains six or more seeds and if nor harvest is possible, the player can decide to play or not the {\it nyumba}.
\paragraph{}\label{NyuGR3}
 All rules for the {\it nyumba} saying ``six or more seeds" do not apply if the {\it nymba} contains less than six seeds, but they reapply if it again gets at least six seeds and has not been lost.

\subsection{\bf The {\it nyumba} during {\it kunamua} stage}\label{NyuTndks} 
Rules to apply besides the general ones.

\subsubsection{}\label{NyuTndks1} 
It is not allowed to start a move without harvest ({\it kutakata}) from the owned {\it nyumba} with six or more seeds.
\subsubsection{}\label{NyuTndks2}
If the {\it nyumba} is the only pit filled at the front row, then one seed from the store must be put in it, two seeds have to be taken out and spread to the left or to the right.
\subsubsection{}\label{NyuTndks3}
If the player still owns the {\it nyumba} and a sowing ends in the {\it nyumba} then
\paragraph{}\label{NyuTndks3.1}
The move ends during {\it kutakata} if the {\it nyumba} contains six or more seeds
\paragraph{}\label{NyuTndks3.2} 
The move ends during a harvest move if no harvest is possible at the {\it nyumba} and if the player wishes to stop

\subsection{\bf The {\it nyumba} during the {\it mtaji} stage.}\label{NyuTndtms}
Rule to apply besides the general ones.

\subsubsection{}\label{NyuTndtms1}
If a {\it nyumba} is still owned during the {\it mtaji} stage, it cannot be {\it kutakatia}-ed (see \ref{Kut1.1}).

\section{Special rules}\label{Sr}

\subsection{\bf Kutakatia}\label{Kut} 
If a player must {\it kutakata}, after the opponent {\it kutakata}-ing has left him/her with an exactly one of the opponent's pit that can be harvested, then the opponent is not allowed to empty it. This is a {\it kutakatia}-ed pit.

\subsubsection{}\label{Kut1}
However, a pit cannot be {\it kutakatia}-ed (e.g., the opponent is allowed to empty if) if it is:
\paragraph{}\label{Kut1.1} 
The still owned {\it nyumba}, or
\paragraph{}\label{Kut1.2}
The only occupied pit in the front row, or
\paragraph{}\label{Kut1.3}
The only it containing more than one seed in the front row.

\subsection{}\label{Kut2}
If {\it kuendelea} ends in a {\it kutakatia}-ed pit, the move ends.
\section*{Appendix}\label{App}
{\bf Notational system} Moves are indicated by the row (`A',`B',`a',`b') and number of the pit from which the move starts (`1'-`8'). The direction of the move is indicated by `$<$' or `$>$'. When a player decides to play the house, a `+' is added to the move. A {\it kutakata} move is indicated by an asterisk `*', a {\it kutakatia} is indicated by two asterisks `**'. In {\it kunamua} stage, the row indication can be omitted. If the capturing pit is a {\it kichwa} or {\it kimbi}, the direction can be omitted.

The direction indicator is relative to the player at move.  It indicates the direction in which the hand moves after putting hte first seed in {\it kunamua} stage or after picking up the seeds of a pit in {\it mtaji} stage.  So, in a capture move during {\it kunamua} stage, the direction indicates whether the left (`$<$') or right (`$>$') {\it kichwa} is chosen to be started from.  A game transcript consists of a head containing the game information and one line for every two piles (one move).  A move line starts with the move number, then a colon, a space, the move for South, a space, the move for North and a semicolon follow.  After the semicolon, comments may be added.

{\it Revised and modified form:\\
\url{https://www.kibao.org/doc/regole2009kiba_en.pdf}\\
which was originally modified from\\ 
\url{www.fdg.unimaas.nl/educ/donkers/games/Bao/rules.html}}

\end{document}